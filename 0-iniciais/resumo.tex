\begin{resumo}

Esta tese investiga o uso de bots como facilitadores de interações naturais em ambientes virtuais de ensino, visando viabilizar metodologias ativas no contexto remoto, considerando que a ausência de interações presenciais pode comprometer a eficácia de abordagens pedagógicas centradas no aluno. É explorado a definição e componentes de bots, suas aplicações educacionais e destaca três princípios fundamentais para interações mediadas: comunicação multidirecional, engajamento ativo e adaptação contextual. O trabalho apresenta o desenvolvimento de um bot educacional integrado ao Discord, plataforma escolhida por suas características que emulam um ambiente interativo, incluindo um dashboard exclusivo para professores que permite controle pedagógico não-invasivo e implementa recursos como feedback em tempo real, atividades colaborativas e ferramentas para aprendizagem baseada em problemas. A implementação técnica utiliza a biblioteca Concord em C, desenvolvida pelo autor, com arquitetura modular que gerencia publicação de conteúdo, interações, análise de dados e persistência. A prova de conceito demonstra funcionalidades como publicação estruturada, mecanismos de feedback rápido e coleta anônima de dúvidas, propondo uma metodologia de avaliação baseada em questionários, logs automáticos e entrevistas para mensurar engajamento, impacto pedagógico, usabilidade e viabilidade de implementação de metodologias ativas, concluindo que bots educacionais podem efetivamente aproximar o ambiente virtual da espontaneidade das interações presenciais, elemento fundamental para o sucesso das metodologias ativas no ensino remoto.

\end{resumo}
