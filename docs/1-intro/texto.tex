\chapter{Introdução}
\label{cap:intro}

O ensino remoto tem se consolidado como uma alternativa viável para a
disseminação do conhecimento, especialmente em cenários que exigem
distanciamento social \cite{fabiane2024}. No entanto, essa modalidade apresenta
desafios significativos, como a manutenção do engajamento dos alunos e a
efetividade da comunicação entre docentes e discentes. A ausência de interações
presenciais pode levar a uma experiência educacional menos dinâmica e
participativa, distanciando as práticas pedagógicas de uma comunicação natural e
espontânea \cite{fabiane2024}.

As metodologias ativas \cite{prince2004} de aprendizagem representam uma
abordagem educacional que coloca o aluno como protagonista do processo de
aprendizagem, em contraste com o ensino tradicional onde o estudante assume um
papel predominantemente passivo. Estas metodologias envolvem participação
direta, reflexão contínua e engajamento prático do aluno na construção do
conhecimento. No contexto presencial, técnicas como aprendizagem baseada em
problemas \cite{yew2016}, sala de aula invertida \cite{vanalten2019} e
aprendizagem colaborativa \cite{laal2012} já demonstraram resultados positivos.
No entanto, sua aplicação em ambientes remotos permanece um desafio
significativo devido às limitações de interação natural entre os participantes
\cite{fabiane2024}.

A integração de tecnologias interativas no ambiente educacional virtual emerge
como elemento essencial para superar os obstáculos do ensino remoto e viabilizar
metodologias ativas neste contexto.

Dentre essas ferramentas capazes de recriar os elementos de metodologias ativas,
estão os \textit{bots}. Os \textit{bots}, programas de computador capazes de
simular interações humanas de forma automatizada e personalizada, apresentam-se
como ferramentas capazes de recriar elementos de naturalidade na comunicação
digital, aproximando o ambiente virtual da espontaneidade característica das
interações presenciais \cite{okonkwo2021}.

Este trabalho tem como objetivo investigar o uso de \textit{bots} como
facilitadores de interações naturais em ambientes virtuais, o que poderia
viabilizar a aplicação de metodologias ativas no ensino remoto. A pesquisa parte
da observação que os \textit{bots} podem atuar como pontes tecnológicas que
diminuem a distância comunicativa entre participantes em ambientes virtuais,
promovendo um fluxo mais natural e espontâneo de interações entre professores e
alunos, sem que isso represente uma sobrecarga adicional para os docentes.

Para alcançar este objetivo, foi desenvolvido um \textit{bot} integrado ao
Discord, uma plataforma de comunicação virtual que, embora não seja
tradicionalmente educacional, foi escolhida por oferecer recursos que emulam
eficientemente um ambiente de ensino interativo. A plataforma suporta
videoconferência, \textit{chat} simultâneo, compartilhamento de conteúdo e
criação de enquetes, proporcionando um ecossistema digital onde interações
naturais podem ser facilitadas por um \textit{bot} que intermedia as interações
entre professor e aluno.

A pesquisa busca analisar como este \textit{bot} pode criar um ambiente virtual
onde interações naturais são facilitadas, tornando viável a aplicação de
princípios de ensino ativo mesmo à distância. O estudo examina especificamente
como esta ferramenta pode transformar a natureza das comunicações digitais
educacionais, aproximando-as da fluidez e espontaneidade das interações
presenciais, elementos fundamentais para o sucesso das metodologias ativas.

A eficácia da ferramenta é avaliada em três dimensões principais: o aumento do
engajamento espontâneo dos alunos durante as aulas remotas, a fluidez e
naturalidade da comunicação entre docentes e discentes mediada pelo
\textit{bot}, e a receptividade dos usuários quanto à integração dessa
tecnologia como elemento natural do processo educacional.

O efeito do uso do \textit{bot} proposto foi avaliado através da coleta de dados
via preenchimento de questionários enviados aos participantes, e de informações
extraídas pelo \textit{bot} em sua mediação das interações básicas durante a
experiência prática.

Este trabalho está organizado da seguinte forma: o Capítulo 2 apresenta a
revisão bibliográfica, abordando conceitos fundamentais sobre \textit{bots},
interações naturais e metodologias ativas em ambientes virtuais; o Capítulo 3
descreve a concepção e implementação do \textit{bot} educacional proposto, com
foco nas funcionalidades que promovem interações naturais; o Capítulo 4 detalha
a prova de conceito e analisa os resultados obtidos no ambiente educacional
remoto; e o Capítulo 5 apresenta a conclusão com análise dos resultados,
limitações do estudo e sugestões de trabalhos futuros.
