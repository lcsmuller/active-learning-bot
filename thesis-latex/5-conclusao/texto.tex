\chapter{Conclusão}
\label{cap:conclusao}

% Usar o graphicspath para buscar figuras no subdiretório figuras
\graphicspath{\currfiledir/figuras/}

A implementação do \textit{bot} desenvolvido com a biblioteca Concord demonstrou
que é possível integrar, de forma natural e não invasiva, tecnologias de
interação ativa ao ambiente de ensino remoto. A ferramenta permitiu que alunos
interagissem com o conteúdo de aula e expressassem \textit{feedbacks}
espontâneos, contribuindo para a adaptação do professor em tempo real.

O Capítulo está organizado da seguinte forma: a Seção \ref{sec:sintese}
apresenta uma síntese dos principais resultados obtidos no experimento,
incluindo os achados relacionados à eficácia na promoção de metodologias ativas
e aspectos técnicos da implementação, a Seção \ref{sec:limitacoes-conclusao} 
discute as limitações identificadas no estudo e suas implicações, a Seção
\ref{sec:contribuicoes} destaca as contribuições do trabalho para o campo da
educação mediada por tecnologia, e finalmente, a Seção \ref{sec:trabalhos-futuros}
apresenta direções para trabalhos futuros que podem expandir e aprimorar os
resultados obtidos.

\section{Síntese dos Resultados}
\label{sec:sintese}

A avaliação experimental do \textit{bot} educacional demonstrou resultados
promissores em múltiplas dimensões pedagógicas e técnicas. Com 10 participantes
atuando tanto como professores quanto como alunos, o experimento forneceu uma
perspectiva abrangente sobre a eficácia da ferramenta em contextos educacionais
remotos.

\subsection{Principais Achados}

\subsubsection{Eficácia na Promoção de Metodologias Ativas}

O objetivo principal do trabalho - facilitar a implementação de metodologias
ativas em ambientes remotos - foi alcançado com sucesso significativo. A
ferramenta obteve média de 4,7 (em escala de 1 a 5) na questão sobre tornar a
aula mais interativa, com 90\% dos participantes atribuindo notas máximas. Este
resultado confirma que o \textit{bot} efetivamente transformou a dinâmica
tradicional de aulas remotas, tradicionalmente caracterizadas por comunicação
unidirecional.

\subsubsection{Redução de Barreiras de Participação}

O anonimato seletivo implementado no sistema mostrou-se um fator crucial para o
engajamento. Os participantes destacaram consistentemente a importância de
"poder comentar em anônimo" e não precisar "responder em áudio", indicando que
a ferramenta efetivamente reduziu inibições comuns em ambientes virtuais de
aprendizagem. Esta funcionalidade atendeu diretamente ao princípio de
"comunicação multidirecional" estabelecido no referencial teórico.

\subsubsection{Melhoria na Comunicação Professor-Aluno}

A facilitação da comunicação obteve uma das melhores avaliações (média 4,7),
confirmando que o sistema conseguiu criar canais eficazes de \textit{feedback}
entre professor e alunos. Comentários indicaram que o \textit{bot} "torna o chat
um canal mais viável para interagir com o professor", abordando uma limitação
real do ensino remoto onde professores frequentemente negligenciam canais
textuais em favor da comunicação por voz.

\subsubsection{Aceitação e Potencial de Adoção}

Com 90\% dos participantes manifestando desejo de usar a ferramenta em mais
aulas (média 4,6), o experimento demonstrou forte aceitação da tecnologia. Este
resultado é particularmente significativo considerando que os participantes
experimentaram tanto os benefícios quanto as limitações atuais da implementação.

\subsection{Limitações Identificadas}

\subsubsection{Interface de Usuário}

A principal barreira identificada foi a dependência de comandos de texto para
interação. Múltiplos participantes solicitaram interfaces mais intuitivas,
especificamente botões clicáveis em lugar de comandos como "/ask" e "/answer".
Esta limitação, embora não comprometa a funcionalidade, pode afetar a adoção em
larga escala.

\subsubsection{Curva de Aprendizado}

Embora a ferramenta tenha sido considerada relativamente fácil de usar (média
4,3), comentários indicaram que alguns aspectos da interface requerem
familiarização. A implementação de interfaces mais visuais poderia reduzir
significativamente esta barreira inicial.

\subsection{Validação dos Objetivos Específicos}

\subsubsection{Integração Não-invasiva}

O objetivo de criar uma integração sutil ao fluxo de trabalho docente foi
parcialmente validado. Embora participantes tenham elogiado a natureza
não-intrusiva da ferramenta, alguns comentários sugeriram que professores
precisam de múltiplas telas para gerenciar simultaneamente o \textit{dashboard}
e materiais de aula, indicando oportunidades de melhoria na ergonomia da
solução.

\subsubsection{Viabilidade Técnica}

A implementação usando a biblioteca Concord em C demonstrou-se tecnicamente
viável, com o sistema operando estável durante todas as sessões experimentais.
A arquitetura dual (\textit{bot} + \textit{dashboard}) atendeu adequadamente às
necessidades de separação entre canal de comando e canal de interação.

\subsubsection{Métricas de Engajamento}

As funcionalidades implementadas (reações, dúvidas anônimas, execução de código
e \textit{quizzes}) foram amplamente utilizadas, com cada uma sendo adotada por
60-70\% dos participantes. Esta distribuição equilibrada indica que o conjunto
de recursos atendeu a diferentes preferências e estilos de interação.

\subsection{Contribuições Científicas Validadas}

Os resultados confirmaram as contribuições científicas propostas:

\begin{enumerate}
\item \textbf{Demonstração de viabilidade}: A integração sutil de tecnologias
interativas no ensino remoto foi comprovadamente viável e eficaz.
\item \textbf{Evidências empíricas}: O experimento forneceu dados quantitativos
e qualitativos robustos sobre o impacto positivo de \textit{bots} educacionais.
\item \textbf{Arquitetura flexível}: A separação entre \textit{dashboard} e
\textit{bot} mostrou-se adequada para diferentes contextos pedagógicos.
\item \textbf{Métricas de avaliação}: As métricas propostas (engajamento,
impacto pedagógico, usabilidade) capturaram adequadamente a eficácia da solução.
\end{enumerate}

Em síntese, o experimento validou a hipótese central de que \textit{bots}
educacionais podem efetivamente facilitar metodologias ativas em ambientes
remotos, embora melhorias na experiência do usuário sejam necessárias para
maximizar o potencial de adoção da tecnologia.

\section{Limitações do Estudo}
\label{sec:limitacoes-conclusao}

A metodologia e implementação deste trabalho apresentam algumas limitações que
devem ser consideradas na interpretação dos resultados e no planejamento de
estudos futuros:

\begin{enumerate}
\item \textbf{Amostra limitada}: O número de participantes e disciplinas pode
não representar adequadamente todos os contextos educacionais, restringindo a
generalização dos resultados.
\item \textbf{Efeito novidade}: O interesse inicial pela tecnologia pode ter
influenciado positivamente os resultados de curto prazo, um fenômeno comum em
estudos de novas tecnologias educacionais.
\item \textbf{Viés de seleção}: Os participantes voluntários podem não
representar adequadamente o perfil completo de alunos e professores,
particularmente no que se refere a diferentes níveis de familiaridade
tecnológica.
\item \textbf{Duração reduzida}: Enquanto um curso real se estende por semanas
ou meses, o experimento teve duração limitada, impossibilitando a observação de
efeitos de longo prazo como adaptação dos usuários, fadiga ou mudanças no
engajamento ao longo do tempo.
\item \textbf{Diversidade contextual}: O experimento foi realizado em um
contexto específico (disciplina de programação em nível de graduação), o que
pode limitar a generalização dos resultados para outros contextos educacionais
que possuem necessidades e dinâmicas distintas.
\item \textbf{Ambiente simulado}: Embora o experimento tenha buscado recriar
condições próximas à realidade, o ambiente de sala de aula simulado pode não ter
reproduzido completamente as dinâmicas e desafios de um curso real, como
discutido na Seção \ref{subsec:desafios}. Os participantes estavam cientes de
que estavam em um experimento, o que pode ter afetado seu comportamento (efeito
\textit{Hawthorne}).
\item \textbf{Avaliação de IHC}: Conforme discutido na Seção \ref{subsec:ihc},
interfaces centradas no usuário exigem avaliações extensivas. O experimento
realizado pode não ter capturado todos os aspectos de usabilidade e experiência
do usuário necessários para uma avaliação completa dos princípios de
\textit{design} centrado no usuário.
\item \textbf{Limitações técnicas}: A implementação em C usando a biblioteca
Concord, embora tenha proporcionado controle granular sobre a \textit{API}, pode
apresentar desafios de manutenção e extensibilidade em comparação com linguagens
de mais alto nível.
\end{enumerate}

Reconhecer estas limitações oferece perspectivas importantes para a
interpretação dos resultados e para o direcionamento de pesquisas futuras neste
campo.

\section{Contribuições do Trabalho}
\label{sec:contribuicoes}

Este trabalho contribui para o campo da educação digital ao:

\begin{enumerate}
\item Demonstrar a viabilidade de integração sutil de tecnologias interativas no
ensino remoto
\item Fornecer evidências empíricas sobre o impacto positivo de \textit{bots} 
educacionais no engajamento
\item Propor uma arquitetura flexível e de baixo impacto para interações
educacionais
\item Estabelecer métricas para avaliação da eficácia de ferramentas de
interação em ambientes virtuais
\end{enumerate}

\section{Trabalhos Futuros}
\label{sec:trabalhos-futuros}

Para trabalhos futuros, recomenda-se:

\begin{enumerate}
\item Expandir a aplicação do \textit{bot} para outros contextos e disciplinas
além da programação
\item Integrar suporte a outras plataformas educacionais (como Moodle, Google
Meet, BBB (\textit{Big Blue Button}), etc.)
\item Explorar o uso de inteligência artificial para tornar os
\textit{feedbacks} ainda mais personalizados e contextuais
\item Realizar estudos longitudinais para avaliar o impacto a longo prazo no
desempenho acadêmico
\item Desenvolver recursos adicionais para análise de padrões de interação e
geração de insights pedagógicos
\item Investigar o potencial do \textit{bot} como ferramenta de avaliação
contínua e formativa
\item Desenvolver \textit{dashboards} com métricas visuais consolidadas de
engajamento e participação dos alunos
\end{enumerate}

A promissora interseção entre tecnologia e educação continua a oferecer
oportunidades para melhorar a experiência de ensino-aprendizagem, especialmente
em contextos remotos, e o presente trabalho busca contribuir com este avanço.
