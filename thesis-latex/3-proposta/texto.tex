\chapter{\textit{Bot} Educacional para uso em Metodologias Ativas em Ambientes
Virtuais}
\label{cap:bot}

% Usar o graphicspath para buscar figuras no subdiretório figuras
\graphicspath{\currfiledir/figuras/}

O Capítulo está organizado da seguinte forma: a Seção \ref{sec:visao}
apresenta a visão conceitual da aplicação e o modelo de interação proposto, a
Seção \ref{sec:integracao} discute a integração do \textit{bot} com o ambiente
educacional e o \textit{dashboard} do professor, a Seção \ref{sec:recursos}
detalha os recursos específicos implementados para promoção de metodologias
ativas, incluindo \textit{feedback} em tempo real e aprendizagem baseada em
problemas, e finalmente, a Seção \ref{sec:exemplo} apresenta um exemplo prático
completo de uso do sistema em uma de programação da disciplina CI1055 (Algoritmos
e Estruturas de Dados I) do DINF da UFPR \cite{ufpr2021ci1055}.

\section{Visão Conceitual da Aplicação}
\label{sec:visao}

Como visto na Seção \ref{sec:metodologia}, o \textit{bot} educacional foi
projetado para atuar como uma ponte entre professor e alunos, promovendo trocas
mais naturais de informações e \textit{feedback}.

A Figura \ref{fig:modelo-interacao} a seguir ilustra o modelo conceitual de
interação entre os participantes do processo educacional mediado pelo
\textit{bot}:

\begin{figure}[htb]
\centering
\includegraphics[width=16cm]{modelo-interacao.png}
\caption{Fluxo de interações no ambiente educacional virtual mediado pelo
\textit{bot}. A figura mostra um diagrama com o professor à esquerda, o
\textit{dashboard} do professor como interface de controle, o \textit{bot}
educacional ao centro, e os alunos à direita, ilustrando os fluxos de
comunicação: (a) professor controlando a aula via \textit{dashboard}, (b)
\textit{dashboard} enviando comandos ao \textit{bot}, (c) \textit{bot} 
processando e disponibilizando o material para os alunos, (d) alunos interagindo
com o conteúdo, (e) \textit{bot} coletando \textit{feedback} dos alunos, e (f)
\textit{bot} fornecendo análises em tempo real ao professor através do
\textit{dashboard}.} 
\label{fig:modelo-interacao}
\end{figure}

O modelo de interação implementado neste trabalho fundamenta-se nos três
princípios para interação mediada por \textit{bots} na educação discutidos na
Seção \ref{subsec:principios}: (1) comunicação multidirecional, (2) engajamento
ativo e (3) adaptação contextual. Estes princípios nortearam todo o processo de
\textit{design} e desenvolvimento da solução, garantindo que o \textit{bot}
efetivamente contribua para a implementação de metodologias ativas no ambiente
virtual.

\section{Integração com o Ambiente Educacional}
\label{sec:integracao}

O \textit{bot} foi projetado para se integrar ao Discord, pelos motivos
discutidos na Seção \ref{sec:ferramentas}. A integração sutil com o ambiente
educacional, refere-se à capacidade do \textit{bot} de participar do processo
educacional sem causar rupturas no fluxo natural da aula ou exigir mudanças
drásticas nas práticas pedagógicas já estabelecidas. Essa sutileza manifesta-se
em três dimensões, (1) presença não-intrusiva, (2) curva de aprendizado reduzida
e (3) flexibilidade metodológica:

\begin{enumerate}
\item \textbf{Presença não-intrusiva}: O \textit{bot} não interrompe a condução
da aula, apenas complementa as atividades quando solicitado ou programado.
\item \textbf{Curva de aprendizado reduzida}: Professores e alunos não precisam
dominar ferramentas complexas, pois as interações ocorrem através de comandos
intuitivos e reações simples.
\item \textbf{Flexibilidade metodológica}: O sistema adapta-se a diferentes
estilos de ensino, não impondo uma abordagem pedagógica específica.
\end{enumerate}

Para materializar esta integração, o sistema também disponibiliza um
\textit{dashboard} específico para uso do professor, como visto em
\ref{subsec:dashboards}, que permite o controle da aula de forma centralizada e
intuitiva, sem a necessidade de comandos complexos ou interrupções no fluxo de
comunicação, como será detalhado na Seção \ref{subsec:dashboard}.

\subsection{\textit{Dashboard} do Professor}
\label{subsec:dashboard}

Um elemento chave do sistema é o \textit{dashboard} exclusivo para o professor,
que permite controlar o fluxo da aula sem a necessidade de inserir comandos no
\textit{chat} principal. Este \textit{dashboard} é apresentado como uma
interface \textit{web}, acessível apenas pelo professor, que se comunica com o
\textit{bot} em tempo real. Esta abordagem está diretamente alinhada com o
objetivo de proporcionar uma integração não-invasiva ao fluxo de trabalho
docente, conforme estabelecido na Seção \ref{sec:objetivos} do Capítulo
\ref{cap:revisao}. Através dele, o professor pode:

\begin{itemize}
\item Visualizar estatísticas de engajamento dos alunos em tempo real
\item Receber alertas sobre dúvidas e dificuldades dos alunos
\item Lançar atividades interativas e acompanhar seu progresso
\item Obter relatórios detalhados sobre o desempenho da turma
\item Destacar respostas e discussões dos alunos para promover a colaboração
\end{itemize}

Esta abordagem permite que o professor mantenha o controle pedagógico da aula
sem interrupções no fluxo da comunicação, enquanto os alunos interagem
diretamente com o \textit{bot} através de comandos e reações no ambiente do
Discord.

\section{Recursos para Promoção de Metodologias Ativas}
\label{sec:recursos}

O \textit{bot} implementa diversos recursos específicos para viabilizar
metodologias ativas em ambiente remoto. Na Seção \ref{subsec:feedback} são
apresentados mecanismos de retorno imediato que permitem ao professor avaliar 
continuamente a compreensão dos alunos, e a Seção \ref{subsec:pbl} trata das 
funcionalidades específicas para resolução de problemas que estimulam o 
raciocínio crítico e a aplicação prática de conceitos.

\subsection{Mecanismos de Retorno Imediato}
\label{subsec:feedback}

Um dos principais desafios do ensino remoto é perceber as reações dos alunos. O
\textit{bot} desenvolvido permite que os estudantes expressem sua compreensão ou
dúvidas durante a explanação, sem interromper o fluxo da aula, através da 
implementação de (1) agregação visual de reações, (2) sistema de notificações, e
(3) canal de comunicação anônima:

\begin{itemize}
\item \textbf{Agregação visual de reações}: Uma interface que coleta e exibe 
de forma consolidada as reações dos alunos sobre sua compreensão do conteúdo
\item \textbf{Sistema de notificações inteligentes}: Alertas automáticos ao 
professor quando um número significativo de alunos indica não compreender um tópico
\item \textbf{Canal de comunicação anônima}: Funcionalidade que permite aos 
alunos enviar questões sem exposição pública de sua identidade
\end{itemize}

A escolha dessas funcionalidades baseou-se na identificação de elementos que
tornam uma conversa por texto interativa e natural. Em ambientes de mensagens
instantâneas, observa-se que a interação fluida depende de mecanismos como
reações rápidas (emojis), possibilidade de comunicação anônima e indicadores
visuais de engajamento. Transportar esses aspectos para o ambiente educacional
remoto visa recriar a naturalidade das interações presenciais, onde o professor
consegue perceber expressões faciais, gestos e reações instantâneas dos
alunos \cite{huang2021}.

\subsection{Funcionalidades para Resolução de Problemas}
\label{subsec:pbl}

Inspirado nos princípios da ``aprendizagem baseada em problemas'' (vide Seção 
\ref{cap:intro}), foram desenvolvidas funcionalidades específicas que estimulam
o raciocínio crítico e a resolução prática de situações-problema. Dentre os 
diversos elementos que poderiam ser implementados, foram selecionados os 
seguintes, priorizando simplicidade de implementação devido ao caráter 
exploratório deste trabalho:

\begin{itemize}
\item \textbf{Sistema de desafios com cronometragem}: Problemas apresentados 
com tempo definido para resolução, criando senso de urgência pedagógica
\item \textbf{Mecanismo de orientação progressiva}: Sugestões e dicas que são 
liberadas gradualmente durante o processo de resolução, conforme a necessidade
\item \textbf{Ambiente de execução remota}: Para disciplinas de programação, um
sistema que permite compilação e execução de códigos submetidos pelos alunos
\item \textbf{Compartilhamento pedagógico de soluções}: Funcionalidade que permite
ao professor selecionar e compartilhar soluções dos alunos de forma anônima, 
promovendo aprendizado colaborativo sem exposição individual
\end{itemize}

A seleção dessas funcionalidades fundamentou-se na observação de como conversas
por texto se tornam mais dinâmicas e envolventes. Elementos como desafios com
tempo limitado simulam a urgência e foco das interações síncronas, enquanto
dicas progressivas reproduzem o aspecto colaborativo de uma conversa onde
informações são reveladas gradualmente conforme a necessidade. A execução
imediata de código, por sua vez, proporciona tanto o \textit{feedback}
instantâneo característico de diálogos interativos quanto um espaço mais
igualitário, onde alunos com recursos computacionais limitados conseguem
executar código remotamente mesmo em máquinas menos potentes ou dispositivos
móveis, eliminando barreiras técnicas que poderiam comprometer sua
participação\cite{fabiane2024}. Por fim, o compartilhamento anônimo de soluções
permite que diferentes abordagens sejam discutidas coletivamente, mantendo o
aspecto pedagógico da aprendizagem colaborativa sem expor individualmente os
alunos, reproduzindo o ambiente de discussão natural de uma sala de aula
presencial.

\section{Exemplo Prático: Aula de Comandos de Repetição}
\label{sec:exemplo}

Para ilustrar a aplicação concreta do \textit{bot} em um contexto educacional
real, apresentamos a seguir um cenário baseado em uma aula da disciplina CI1055
- Algoritmos e Estruturas de Dados I, ministrada no Departamento de Informática
da UFPR \cite{ufpr2021ci1055}. O exemplo demonstra como o \textit{bot} auxilia o
professor durante uma aula sobre ``Comandos de Repetição'' em Pascal. 

O cenário completo seguiu um fluxo de interação que exemplifica como as 
funcionalidades descritas nas seções anteriores se articulam na prática:

\begin{enumerate}
\item \textbf{Professor prepara material via \textit{dashboard}}, organizando conteúdo e atividades;
\item \textbf{\textit{Bot} então publica conteúdo formatado no Discord}, iniciando a interação;
\item \textbf{Alunos começam a interagir com reações e comandos}, participando ativamente;
\item \textbf{Professor recebe \textit{feedback} em tempo real}, adaptando a aula conforme necessário;
\item \textbf{Sistema gera relatório automático pós-aula}, fechando o ciclo pedagógico.
\end{enumerate}

Nas próximas seções, detalharemos as etapas de preparação da aula pelo professor 
e as interações que ocorrem durante a sessão síncrona, evidenciando como os 
recursos do \textit{bot} facilitam a implementação das metodologias ativas.

\subsection{Preparação da Aula}
\label{subsec:preparacao}

Antes da aula, o professor utiliza o \textit{dashboard} para preparar o material
didático:

\begin{lstlisting}[
  basicstyle=\ttfamily\footnotesize,
  backgroundcolor=\color{gray!10},
  breaklines=true,
  captionpos=b,
  commentstyle=\color{green!50!black},
  frame=single,
  keywordstyle=\color{blue},
  numbers=left,
  numbersep=5pt,
  numberstyle=\tiny\color{gray},
  stringstyle=\color{red},
  showstringspaces=false,
  tabsize=2
]
[(*@\textit{Dashboard}@*) do Professor]
> Criar Nova Aula Por Videoconferência
Título: "Comandos de Repetição em Pascal"
Código de Disciplina: "AED1-2024"
Descrição: "Introdução aos comandos de repetição em Pascal com foco no comando while"
Tópicos: "Loops", "Comando while", "Repetição", "Pascal"

> Adicionar Conteúdo
[Título] "Objetivos da aula"
[Conteúdo] "Introduzir conceitos de repetição, apresentar o comando while, 
           resolver exemplos práticos"

> Adicionar Conteúdo
[Título] "Exemplo inicial: imprimir números de 1 a 5"
[Conteúdo] 
```
program imprimir_de_1_a_5;
begin
  writeln(1);
  writeln(2);
  writeln(3);
  writeln(4);
  writeln(5);
end.
```
[Tipo] Código Pascal

> Configurar Quiz
[Pergunta] "Ao incrementar uma variável dentro de um loop while, 
           qual operação utilizamos em Pascal?"
[Opções] 
- "i := i + 1" (CORRETA)
- "i++"
- "i += 1"
- "increment(i)"
[Tempo] 60 segundos
\end{lstlisting}

\subsection{Interação Durante a Aula}
\label{subsec:interacao}

Durante a aula síncrona, as seguintes interações ocorrem:

\begin{lstlisting}[
  basicstyle=\ttfamily\footnotesize,
  backgroundcolor=\color{gray!10},
  breaklines=true,
  captionpos=b,
  commentstyle=\color{green!50!black},
  frame=single,
  keywordstyle=\color{blue},
  numbers=left,
  numbersep=5pt,
  numberstyle=\tiny\color{gray},
  stringstyle=\color{red},
  showstringspaces=false,
  tabsize=2
]
[(*@\textit{Dashboard}@*) do Professor]
> Iniciar Aula "Comandos de Repetição em Pascal"
Sistema: Canal de videoconferência #AED1-2024 criado para a aula.

[Discord - Canal #AED1-2024]
Bot: @everyone O professor iniciou a aula "Comandos de Repetição em Pascal". 
     Entre no canal de videoconferência para confirmar sua presença.

[Vários alunos entram no canal de vídeoconferência]

Professor [Canal #AED1-2024]: Vamos começar entendendo por que precisamos de
      comandos de repetição. Observem este exemplo inicial no canal.

[(*@\textit{Dashboard}@*) do Professor]
> Mostrar Código "Exemplo inicial"
Sistema: Código exibido no canal da disciplina #AED1-2024:

[Discord - Canal #AED1-2024]
Bot: 
```pascal
program imprimir_de_1_a_5;
begin
  writeln(1);
  writeln(2);
  writeln(3);
  writeln(4);
  writeln(5);
end.
```
Para testar este código, utilize /execute

[(*@\textit{Dashboard}@*) do Professor]
> Iniciar Discussão
[Pergunta] "Qual o problema desta abordagem se quisermos imprimir de 1 até 1000?"

[Discord - Canal #AED1-2024]
Bot: DISCUSSÃO: Qual o problema desta abordagem se quisermos imprimir de 1 até 1000?
     Use (*@\textit{/answer}@*) para participar da discussão.

Aluno1: (*@\textit{/answer}@*) Teríamos que escrever mil linhas de código!
Aluno2: (*@\textit{/answer}@*) Código muito repetitivo e difícil de manter.

[(*@\textit{Dashboard}@*) do Professor - Painel de Engajamento]
Status: 15/23 alunos responderam
Participação ativa: 65%
Respostas mais comuns: "código repetitivo" (60%), "muitas linhas" (27%)

[Alguns alunos usam reações no Discord]
[5 alunos reagem com "joinha" (entendi)]
[2 alunos reagem com "?" (tenho dúvida)]

[(*@\textit{Dashboard}@*) do Professor - Alertas]
ATENÇÃO 2 alunos indicaram dúvidas sobre o conceito atual.
Recomendação: Revisitar o conceito com uma abordagem alternativa.

Professor [Canal #AED1-2024]: Estou vendo que temos algumas dúvidas. 
                              Vamos revisitar o conceito de forma diferente.

[(*@\textit{Dashboard}@*) do Professor]
> Mostrar Exemplo Interativo
[Título] "Loop while básico"
[Código]
```pascal
program exemplo;
var i: integer;
begin
  i := 1;
  while i <= 5 do
  begin
    writeln(i);
    i := i + 1;
  end;
end.
```
[Opções] Ativar execução por alunos

[Discord - Canal #AED1-2024]
Bot: EXEMPLO INTERATIVO: Loop while básico
```pascal
program exemplo;
var i: integer;
begin
  i := 1;
  while i <= 5 do
  begin
    writeln(i);
    i := i + 1;
  end;
end.
```
Use /execute para ver o resultado deste código.

Aluno5: /execute
Bot: 
```
1
2
3
4
5
```

Aluno8: /ask O que acontece se eu esquecer de incrementar i dentro do loop?
Bot: @Professor Dúvida enviada anonimamente: "O que acontece se eu esquecer de incrementar i dentro do loop?"

[(*@\textit{Dashboard}@*) do Professor]
> Responder Dúvida
[Criar Exemplo] "Loop infinito"
```pascal
program loop_infinito;
var i: integer;
begin
  i := 1;
  while i <= 5 do
  begin
    writeln(i);
    // i não é incrementado
  end;
end.
```

[Discord - Canal #AED1-2024]
Bot: Resposta à dúvida: O que acontece se esquecer de incrementar i
```pascal
program loop_infinito;
var i: integer;
begin
  i := 1;
  while i <= 5 do
  begin
    writeln(i);
    // i não é incrementado
  end;
end.
```
Aviso: O código acima contém um loop infinito. A execução seria interrompida após 
repetir o mesmo valor várias vezes:
```
1
1
1
1
...
```

[(*@\textit{Dashboard}@*) do Professor]
> Criar Desafio
[Título] "Números pares"
[Descrição] "Escreva um programa que leia um número n e imprima todos os números pares de 2 até n"
[Tempo] 10 minutos
[Testes Automáticos] 
- Entrada: "10" | Saída esperada: "2 4 6 8 10"
- Entrada: "7" | Saída esperada: "2 4 6"

[Discord - Canal #AED1-2024]
Bot: NOVO DESAFIO: Números pares
Escreva um programa que leia um número n e imprima todos os números pares de 2 até n.
Tempo: 10 minutos
Use /solve para começar a resolver.

[Vários alunos utilizam o comando /solve, recebendo acesso a canais privados para resolver o problema]

[(*@\textit{Dashboard}@*) do Professor - Monitor de Progresso]
15/23 alunos iniciaram o desafio
8/23 já submeteram soluções
6/23 passaram em todos os testes

[Após o tempo definido]
[(*@\textit{Dashboard}@*) do Professor]
> Encerrar Desafio
> Compartilhar Solução Anônima [ID: 3]

[Discord - Canal #AED1-2024]
Bot: O tempo para o desafio "Números pares" encerrou!
Estatísticas:
- 19/23 alunos participaram
- 15/19 completaram o desafio com sucesso

Solução compartilhada (anônima):
```pascal
program numeros_pares;
var i, n: integer;
begin
  read(n);
  i := 2;
  while i <= n do
  begin
    writeln(i);
    i := i + 2;
  end;
end.
```

[(*@\textit{Dashboard}@*) do Professor]
> Encerrar Aula
> Gerar Relatório

[Discord - Canal #AED1-2024]
Bot: Esta aula foi encerrada. Um resumo das atividades foi enviado ao professor.
Sistema: Canal de videoconferência #AED1-2024 encerrado.

[(*@\textit{E-mail}@*) do Professor - Relatório Automático]
Resumo da aula: "Comandos de Repetição em Pascal"
- Participação: 23/25 alunos ativos (92%)
- Média de interações por aluno: 7.2
- Tópicos com mais dúvidas: "loop infinito" (5 menções), "incremento de variáveis" (3 menções)
- Desafio "Números pares": 19/23 participaram, 15/19 completaram com sucesso
- Alunos com participação abaixo do esperado: 2 (lista anexa)
\end{lstlisting}
