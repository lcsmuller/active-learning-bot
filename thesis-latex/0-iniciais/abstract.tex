\begin{abstract}

This undergraduate thesis investigates the use of bots as facilitators of
natural interactions in virtual learning environments, aiming to enable active
methodologies in remote teaching contexts. The study is based on the premise
that the absence of face-to-face interactions can compromise the effectiveness
of student-centered pedagogical approaches. It explores the definition and
components of bots, their educational applications, and highlights three
fundamental principles for mediated interactions: multidirectional
communication, active engagement, and contextual adaptation. The work presents
the development of an educational bot integrated with Discord, a platform chosen
for its features that emulate an interactive environment, including an exclusive
dashboard for teachers that allows non-invasive pedagogical control and
implements resources such as real-time feedback, collaborative activities, and
tools for problem-based learning. The technical implementation uses the Concord
library in C, developed by the author, with a modular architecture that manages
content publication, interactions, data analysis, and persistence. The proof of
concept demonstrates functionalities such as structured publication, quick
feedback mechanisms, and anonymous question collection. It is proposed an
evaluation methodology based on questionnaires, automatic logs, and interviews
to measure engagement, pedagogical impact, usability, and feasibility of
implementing active methodologies. It is concluded that educational bots can
effectively bring the virtual environment closer to the spontaneity of
face-to-face interactions, a fundamental element for the success of active
methodologies in remote teaching.

\end{abstract}
