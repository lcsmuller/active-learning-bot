\chapter{Introdução}
\label{cap:intro}

Este capítulo apresenta o contexto e a motivação para o desenvolvimento de um
\textit{bot} educacional que facilite a implementação de metodologias ativas em
ambientes de ensino remoto.

O capítulo está organizado da seguinte forma: a Seção \ref{sec:contexto-intro}
apresenta o contexto do ensino remoto e das metodologias ativas, estabelecendo
a relevância das tecnologias interativas neste cenário, a Seção
\ref{sec:motivacao} discute a motivação de pesquisa, detalhando os
desafios específicos do ensino remoto e a questão central investigada, a Seção
\ref{sec:objetivos} define os objetivos geral e específicos do trabalho, a Seção
\ref{sec:metodologia-intro} descreve a metodologia adotada para o desenvolvimento
e validação da solução, a Seção \ref{sec:contribuicoes-intro} apresenta as
principais contribuições esperadas do trabalho, e finalmente, a Seção
\ref{sec:organizacao} descreve a organização dos demais capítulos.

\section{Contexto}
\label{sec:contexto-intro}

O ensino remoto tem se consolidado como uma alternativa viável para a
disseminação do conhecimento, especialmente em cenários que exigem
distanciamento social \cite{fabiane2024}. Esta modalidade de ensino,
impulsionada por necessidades específicas e pelo avanço das tecnologias
digitais, tornou-se uma realidade permanente na educação contemporânea, exigindo
adaptações significativas nas práticas pedagógicas tradicionais.

As metodologias ativas de aprendizagem \cite{prince2004} representam uma
abordagem educacional que coloca o aluno como protagonista do processo de
aprendizagem, em contraste com o ensino tradicional onde o estudante assume um
papel predominantemente passivo. Estas metodologias envolvem participação
direta, reflexão contínua e engajamento prático do aluno na construção do
conhecimento. No contexto presencial, técnicas como aprendizagem baseada em
problemas \cite{yew2016}, sala de aula invertida \cite{vanalten2019} e
aprendizagem colaborativa \cite{laal2012} já demonstraram resultados positivos.
No entanto, sua aplicação em ambientes remotos permanece um desafio
significativo devido às limitações de interação natural entre os participantes
\cite{fabiane2024}.

A integração de tecnologias interativas no ambiente educacional virtual emerge
como elemento essencial para superar os obstáculos do ensino remoto e viabilizar
metodologias ativas neste contexto. Dentre essas ferramentas, destacam-se os
\textit{bots}, programas de computador capazes de simular interações humanas de
forma automatizada e personalizada, que se apresentam como soluções capazes de
recriar elementos de naturalidade na comunicação digital, aproximando o ambiente
virtual da espontaneidade característica das interações presenciais
\cite{okonkwo2021}.

\section{Motivação}
\label{sec:motivacao}

O ensino remoto, apresenta desafios que limitam sua eficácia educacional. A
ausência de interações presenciais pode levar a uma experiência educacional
menos dinâmica e participativa, distanciando as práticas pedagógicas de uma
comunicação natural e espontânea \cite{fabiane2024}.

Em aulas remotas tradicionais, observa-se frequentemente um padrão de comunicação
unidirecional, onde o professor transmite o conteúdo enquanto os alunos assumem
postura predominantemente passiva \cite{hodges2020}. As interações tendem a ser
limitadas a momentos específicos, como sessões de perguntas ao final da aula, ou
através de canais assíncronos como \textit{e-mails} e fóruns \cite{fabiane2024}.
Este modelo apresenta barreiras significativas à implementação de metodologias
ativas, que dependem de ciclos rápidos de \textit{feedback} e participação
constante dos estudantes \cite{prince2004}.

A pesquisa parte da observação que os \textit{bots} podem atuar como pontes
tecnológicas que diminuem a distância comunicativa entre participantes em
ambientes virtuais, promovendo um fluxo mais natural e espontâneo de interações
entre professores e alunos, sem que isso represente uma sobrecarga adicional
para os docentes. \cite{winkler2018,okonkwo2021,zawacki2019}

A questão de pesquisa central deste trabalho é: \textit{Como um \textit{bot}
educacional pode facilitar a implementação de metodologias ativas em ambientes
de ensino remoto, promovendo interações naturais entre professores e alunos?}

\section{Objetivos do Trabalho} \label{sec:objetivos}

Com base na motivação apresentada, este trabalho tem como objetivo geral
\textbf{desenvolver e avaliar um \textit{bot} educacional assistivo e
conversacional} para plataformas de colaboração, que facilite a implementação de
metodologias ativas em ambientes de ensino remoto \cite{prince2004,okonkwo2021}.

Os objetivos específicos são:

\begin{enumerate}
\item \textbf{Desenvolver um \textit{bot} educacional} que incorpore os três
princípios fundamentais para interação mediada: comunicação multidirecional, 
engajamento ativo e adaptação contextual \cite{seering2018}.
\item \textbf{Implementar funcionalidades específicas} que abordem diretamente
os desafios do ensino remoto \cite{fabiane2024}, com foco especial em manter o 
engajamento dos alunos, facilitar o \textit{feedback} imediato e promover 
interações sociais significativas em ambientes virtuais \cite{han2022}.
\item \textbf{Proporcionar uma integração não-invasiva} da ferramenta ao fluxo
de trabalho docente, aplicando os princípios de \textit{design} centrado no
usuário \cite{roy1987,norman2013} e minimizando a carga cognitiva adicional
\cite{sweller2011}.
\item \textbf{Criar uma prova de conceito funcional} utilizando tecnologias
adequadas ao contexto educacional, considerando aspectos de eficiência,
portabilidade e manutenção \cite{kernighan1988}.
\item \textbf{Estabelecer uma metodologia de avaliação experimental} com
participantes reais assumindo os papéis de professor e alunos, onde o professor
utiliza o \textit{dashboard} de controle \cite{verbert2013} enquanto os alunos 
interagem com o \textit{bot} em um ambiente de sala de aula simulado, permitindo
mensurar a eficácia da solução em situações próximas ao uso real, combinando 
métricas quantitativas e qualitativas.
\end{enumerate}

\section{Metodologia}
\label{sec:metodologia-intro}

Para atingir os objetivos propostos, este trabalho adota uma abordagem
metodológica que combina desenvolvimento tecnológico com validação experimental.
A metodologia está estruturada em quatro etapas principais: revisão
bibliográfica, desenvolvimento da solução, validação experimental e análise de
resultados.

\subsection{Revisão Bibliográfica}
\label{subsec:revisao-metodologia}

Realização de uma revisão sistemática da literatura sobre \textit{bots}
educacionais, metodologias ativas em ambientes virtuais, e tecnologias para
desenvolvimento de sistemas interativos. Esta etapa visa estabelecer o
referencial teórico necessário para fundamentar as escolhas técnicas e
pedagógicas do trabalho.

\subsection{Desenvolvimento da Solução}
\label{subsec:desenvolvimento-metodologia}

O desenvolvimento da solução é baseado na escolha da plataforma Discord,
adequada ao caráter exploratório deste trabalho por suas características
técnicas que emulam eficientemente um ambiente de ensino interativo. A
plataforma oferece recursos essenciais para metodologias ativas: suporte nativo
a videoconferência e \textit{chat} simultâneo (permitindo comunicação
multidirecional), sistema de reações com emojis para \textit{feedback} rápido,
funcionalidades de compartilhamento de conteúdo com formatação rica, capacidade
de criação de enquetes e atividades interativas, e \textit{API} robusta para
integração de \textit{bots} personalizados. Esta escolha permite uma avaliação
inicial dos conceitos propostos em um ambiente controlado e familiar aos
participantes.

O desenvolvimento técnico utiliza a biblioteca Concord em C \cite{muller},
aproveitando suas vantagens em termos de controle granular sobre a \textit{API}
do Discord. A arquitetura do sistema é composta por dois componentes principais:
o \textit{bot} educacional responsável pela interação direta com os alunos, e o
\textit{dashboard} exclusivo para o professor, que permite controlar e monitorar
as atividades do \textit{bot}, visualizar métricas de engajamento e adaptar as
estratégias pedagógicas em tempo real.

\subsection{Validação Experimental}
\label{subsec:validacao-metodologia}

A validação da solução é realizada através de um experimento controlado com
participantes reais assumindo os papéis de professor e alunos. O experimento
conta com 10 participantes (8 da área da informática e 2 da área de humanas),
totalizando 20 sessões individuais, onde cada participante atua uma vez como
aluno e uma vez como professor. O experimento simula um ambiente de sala de
aula remota onde o professor utiliza o \textit{dashboard} de controle enquanto
os alunos interagem com o \textit{bot} durante uma sessão educacional.

A eficácia da ferramenta é avaliada em (1) engajamento dos alunos, (2)
eficácia pedagógica, (3) usabilidade da ferramenta e (4) aceitação da
tecnologia.

\subsection{Análise de Resultados}
\label{subsec:analise-metodologia}

Os dados são coletados através do preenchimento de questionários e comentários
livres pelos participantes, a fim de avaliar aspectos quantitativos e
qualitativos, respectivamente. Dessa maneira a análise busca identificar padrões
recorrentes nas respostas, fornecendo insights sobre a eficácia da solução
proposta e direcionamentos para futuras melhorias.

\section{Contribuições}
\label{sec:contribuicoes-intro}

Este trabalho se diferencia dos esforços anteriores por seu foco específico em 
facilitar metodologias ativas em ambientes remotos \cite{prince2004,yew2016,
vanalten2019,laal2012}, com ênfase na integração não-invasiva à prática docente 
e sua metodologia de avaliação que simula condições reais de uso. Enquanto 
outros trabalhos têm explorado \textit{bots} para responder dúvidas ou fornecer 
\textit{feedback} automatizado \cite{hien2018,demetriadis2018,winkler2018}, esta 
proposta busca transformar a própria dinâmica de interação durante as aulas 
síncronas, com uma avaliação sistemática em um ambiente controlado.

As principais contribuições esperadas incluem a proposição de um
\textbf{modelo conceitual} de interação mediada por \textit{bots} para ambientes
educacionais remotos, fundamentado nos princípios de comunicação
multidirecional, engajamento ativo e adaptação contextual. Além disso, o
trabalho oferece uma \textbf{implementação prática} através do desenvolvimento
de uma solução funcional que demonstra a viabilidade técnica da abordagem,
integrando um \textit{bot} educacional com um \textit{dashboard} de controle
para professores. A \textbf{validação experimental} estabelece uma metodologia
de avaliação robusta que pode ser replicada em outros contextos educacionais,
fornecendo diretrizes para futuras pesquisas na área. Por fim, o estudo
contribui com \textbf{insights pedagógicos} importantes sobre os fatores que
influenciam a eficácia de \textit{bots} educacionais na promoção de metodologias
ativas, oferecendo uma compreensão aprofundada dos desafios e oportunidades
dessa abordagem tecnológica no contexto educacional.

\section{Organização do Trabalho}
\label{sec:organizacao}

Este trabalho está organizado da seguinte forma: o Capítulo 2 apresenta a
revisão bibliográfica, abordando conceitos fundamentais sobre \textit{bots},
interações naturais e metodologias ativas em ambientes virtuais; o Capítulo 3
descreve a concepção e implementação do \textit{bot} educacional proposto, com
foco nas funcionalidades que promovem interações naturais; o Capítulo 4 detalha
a prova de conceito e analisa os resultados obtidos no ambiente educacional
remoto; e o Capítulo 5 apresenta a conclusão com análise dos resultados,
limitações do estudo e sugestões de trabalhos futuros.
