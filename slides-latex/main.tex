\documentclass[aspectratio=169]{beamer}
\usepackage[utf8]{inputenc} % codificacao de caracteres
\usepackage[T1]{fontenc}    % codificacao de fontes
\usepackage[brazil]{babel}  % idioma
\usepackage{graphicx}       %fundo
\usepackage{amsmath}        % matematica
\usepackage{amsfonts}       % fontes matematicas
\usepackage{amssymb}        % simbolos matematicos
\usetheme{default}          % tema
\usecolortheme{orchid}     % cores
\usefonttheme[onlymath]{serif} % fonte modo matematico

\beamertemplatenavigationsymbolsempty % Desativando os botoes de navegacao

% Tela cheia
\hypersetup{pdfpagemode=FullScreen}

% Titulo
\makeatletter
\newcommand\titlegraphicii[1]{\def\inserttitlegraphicii{#1}}
\titlegraphicii{}
\setbeamertemplate{title page}
{
  \vbox{}
   {\usebeamercolor[fg]{titlegraphic}\inserttitlegraphic\hfill\inserttitlegraphicii\par}
  \begin{centering}
    \begin{beamercolorbox}[sep=2pt,center]{institute}
      \usebeamerfont{institute}\insertinstitute
    \end{beamercolorbox}
    \begin{beamercolorbox}[sep=12pt,center]{title}
      \usebeamerfont{title}\inserttitle\par%
      \ifx\insertsubtitle\@empty%
      \else%
        \vskip0.2em%
        {\usebeamerfont{subtitle}\usebeamercolor[fg]{subtitle}\insertsubtitle\par}%
      \fi%     
    \end{beamercolorbox}%
    \vskip0.2em\par
    \begin{beamercolorbox}[sep=2pt,center]{author}
      \usebeamerfont{author}\insertauthor
    \end{beamercolorbox}
    \begin{beamercolorbox}[sep=4pt,center]{date}
      \usebeamerfont{date}\insertdate
    \end{beamercolorbox}%\vskip0.5em
  \end{centering}
  %\vfill
}
\makeatother
\title{\textit{Bot} Educacional para Metodologias Ativas no Ensino Remoto}
%\subtitle{Subtítulo}
\author{\textbf{Lucas Müller}}
\institute{UNIVERSIDADE FEDERAL DO PARANÁ \\ 
SETOR DE CIÊNCIAS EXATAS \\
DEPARTAMENTO DE INFORMÁTICA \\
BACHARELADO EM CIÊNCIA DA COMPUTAÇÃO} % opcional
\date{Orientador: Prof. Bruno Müller Junior}



\AtBeginSubsection[]
{
  \begin{frame}<beamer>{Outline}
    \tableofcontents[currentsection,currentsubsection]
  \end{frame}
}

% Let's get started
\begin{document}
{%
 \usebackgroundtemplate{
  \centering
  \includegraphics[width=\paperwidth]{CapaUFPR.png}
 }
\begin{frame}
  \titlepage
\end{frame}
}
{%
 \usebackgroundtemplate{
  \centering
  \includegraphics[width=\paperwidth]{FundoUFPR2.png}
 }
\begin{frame}{Outline}
  \tableofcontents
  % You might wish to add the option [pausesections]
\end{frame}

% Section and subsections will appear in the presentation overview
% and table of contents.
\section{Introdução e Motivação}

\begin{frame}{Contexto}
  \begin{itemize}
  \item O ensino remoto se consolidou como alternativa viável para educação
  \item Desafios: manutenção do engajamento e comunicação efetiva
  \item Ausência de interações presenciais resulta em experiência menos dinâmica
  \item Metodologias ativas colocam o aluno como protagonista
  \item \textbf{Problema}: Como aplicar metodologias ativas no ensino remoto?
  \end{itemize}
\end{frame}

\begin{frame}{Motivação}
  \begin{itemize}
  \item Metodologias ativas são eficazes no ensino presencial:
    \begin{itemize}
    \item Aprendizagem baseada em problemas
    \item Sala de aula invertida
    \item Aprendizagem colaborativa
    \end{itemize}
  \item Limitações no ambiente remoto:
    \begin{itemize}
    \item Comunicação unidirecional
    \item Falta de feedback imediato
    \item Menor espontaneidade nas interações
    \end{itemize}
  \item \textbf{Solução proposta}: Uso de \textit{bots} como mediadores
  \end{itemize}
\end{frame}

\section{Fundamentação Teórica}

\begin{frame}{O que são \textit{Bots} Educacionais?}
  \begin{itemize}
  \item Programas automatizados que simulam interações humanas
  \item \textbf{Componentes principais}:
    \begin{itemize}
    \item Interface do usuário
    \item Compreensão de linguagem natural
    \item Gerenciador de diálogo
    \item Integração com backend
    \item Geração de resposta
    \end{itemize}
  \item Podem facilitar interações naturais em ambientes virtuais
  \item Potencial para diminuir distância comunicativa no ensino remoto
  \end{itemize}
\end{frame}

\begin{frame}{Princípios para Interação Mediada}
  \begin{block}{Três princípios fundamentais}
  \begin{enumerate}
  \item \textbf{Comunicação multidirecional}: Fluxo bidirecional de informações
  \item \textbf{Engajamento ativo}: Participação contínua dos estudantes
  \item \textbf{Adaptação contextual}: Ajuste em tempo real às necessidades
  \end{enumerate}
  \end{block}
  
  \begin{itemize}
  \item Esses princípios nortearam o desenvolvimento da solução
  \item Objetivo: aproximar ambiente virtual da espontaneidade presencial
  \end{itemize}
\end{frame}

\section{Proposta e Implementação}

\begin{frame}{Arquitetura da Solução}
  \begin{itemize}
  \item \textbf{Plataforma}: Discord (comunicação interativa)
  \item \textbf{Componentes}:
    \begin{itemize}
    \item \textit{Bot} educacional (interação com alunos)
    \item \textit{Dashboard} do professor (controle pedagógico)
    \end{itemize}
  \item \textbf{Implementação}: Biblioteca Concord em C
  \item \textbf{Arquitetura modular}:
    \begin{itemize}
    \item Módulo de publicação
    \item Módulo de interação
    \item Módulo de análise
    \item Módulo de persistência
    \end{itemize}
  \end{itemize}
\end{frame}

\begin{frame}{Funcionalidades Implementadas}
  \begin{columns}
  \begin{column}{0.5\textwidth}
    \textbf{Para os Alunos:}
    \begin{itemize}
    \item Feedback rápido via reações
    \item Dúvidas anônimas
    \item Execução de código
    \item Atividades interativas
    \item Quizzes temporizados
    \end{itemize}
  \end{column}
  \begin{column}{0.5\textwidth}
    \textbf{Para o Professor:}
    \begin{itemize}
    \item Dashboard de controle
    \item Barômetro de compreensão
    \item Sistema de alertas
    \item Relatórios em tempo real
    \item Gerenciamento de atividades
    \end{itemize}
  \end{column}
  \end{columns}
\end{frame}

\begin{frame}{Exemplo Prático: Aula de Programação}
  \begin{itemize}
  \item \textbf{Disciplina}: CI1055 - Algoritmos e Estruturas de Dados I
  \item \textbf{Tópico}: Comandos de Repetição em Pascal
  \item \textbf{Fluxo da aula}:
    \begin{enumerate}
    \item Professor prepara material via dashboard
    \item Bot publica conteúdo formatado
    \item Alunos interagem com reações e comandos
    \item Professor recebe feedback em tempo real
    \item Execução de código pelos alunos
    \item Relatório automático pós-aula
    \end{enumerate}
  \end{itemize}
\end{frame}

\section{Validação e Resultados}

\begin{frame}{Metodologia de Avaliação}
  \begin{itemize}
  \item \textbf{Participantes}: 10 usuários (professores e alunos)
  \item \textbf{Métodos de coleta}:
    \begin{itemize}
    \item Questionários estruturados
    \item Logs automáticos de interação
    \item Observação das sessões
    \end{itemize}
  \item \textbf{Dimensões avaliadas}:
    \begin{itemize}
    \item Engajamento dos alunos
    \item Eficácia pedagógica
    \item Usabilidade da ferramenta
    \item Aceitação da tecnologia
    \end{itemize}
  \end{itemize}
\end{frame}

\begin{frame}{Principais Resultados}
  \begin{itemize}
  \item \textbf{Eficácia na promoção de metodologias ativas}:
    \begin{itemize}
    \item Média 4,7/5,0 para "tornar aula mais interativa"
    \item 90\% dos participantes deram notas máximas
    \end{itemize}
  \item \textbf{Redução de barreiras de participação}:
    \begin{itemize}
    \item Anonimato seletivo bem avaliado
    \item Participantes destacaram não precisar "responder em áudio"
    \end{itemize}
  \item \textbf{Melhoria na comunicação}:
    \begin{itemize}
    \item Média 4,7/5,0 para facilitação da comunicação
    \item "Torna o chat um canal mais viável"
    \end{itemize}
  \end{itemize}
\end{frame}

\begin{frame}{Aceitação e Limitações}
  \begin{block}{Aceitação}
  \begin{itemize}
  \item 90\% desejam usar em mais aulas (média 4,6/5,0)
  \item Forte aceitação da tecnologia
  \item Integração não-invasiva ao fluxo da aula
  \end{itemize}
  \end{block}
  
  \begin{block}{Limitações Identificadas}
  \begin{itemize}
  \item Dependência de comandos de texto
  \item Necessidade de interfaces mais intuitivas
  \item Curva de aprendizado inicial
  \item Necessidade de múltiplas telas para professores
  \end{itemize}
  \end{block}
\end{frame}

\section{Conclusões}

\begin{frame}{Contribuições do Trabalho}
  \begin{itemize}
  \item \textbf{Prova de conceito} de bot educacional funcional
  \item \textbf{Validação empírica} da eficácia em metodologias ativas
  \item \textbf{Arquitetura modular} replicável em outros contextos
  \item \textbf{Princípios de design} para interações mediadas
  \item \textbf{Redução de barreiras} de participação no ensino remoto
  \item \textbf{Dashboard integrado} para controle pedagógico
  \end{itemize}
\end{frame}

\begin{frame}{Trabalhos Futuros}
  \begin{itemize}
  \item \textbf{Interface mais intuitiva}:
    \begin{itemize}
    \item Implementação de botões clicáveis
    \item Redução da dependência de comandos de texto
    \end{itemize}
  \item \textbf{Expansão de funcionalidades}:
    \begin{itemize}
    \item Integração com LMS tradicionais
    \item Suporte a outras linguagens de programação
    \item Análise de sentimentos em tempo real
    \end{itemize}
  \item \textbf{Estudos longitudinais}:
    \begin{itemize}
    \item Avaliação em semestres completos
    \item Análise de impacto no aprendizado
    \end{itemize}
  \end{itemize}
\end{frame}

\section*{Considerações Finais}

\begin{frame}{Resumo das Contribuições}
  \begin{itemize}
  \item
    \alert{Bot educacional} integrado ao Discord demonstrou viabilidade técnica e pedagógica
  \item
    \alert{Metodologias ativas} podem ser efetivamente implementadas no ensino remoto
  \item
    \alert{Interações naturais} foram facilitadas através dos três princípios propostos
  \end{itemize}
  
  \begin{itemize}
  \item
    Próximos passos:
    \begin{itemize}
    \item
      Melhorar interfaces de usuário
    \item
      Expandir para outras disciplinas
    \item
      Estudos longitudinais de impacto
    \end{itemize}
  \end{itemize}
\end{frame}

\begin{frame}{Agradecimentos}
  \begin{center}
  \Large
  Obrigado pela atenção!
  
  \vspace{1cm}
  
  \normalsize
  \textbf{Lucas Müller}\\
  Orientador: Prof. Bruno Müller Junior\\
  
  \vspace{0.5cm}
  
  Universidade Federal do Paraná\\
  Departamento de Informática\\
  Bacharelado em Ciência da Computação
  
  \vspace{1cm}
  
  \textit{Perguntas?}
  \end{center}
\end{frame}



% All of the following is optional and typically not needed. 
\appendix
\section<presentation>*{\appendixname}
\subsection<presentation>*{For Further Reading}

\begin{frame}[allowframebreaks]
  \frametitle<presentation>{For Further Reading}
    
  \begin{thebibliography}{10}
    
  \beamertemplatebookbibitems

  \bibitem{prince2004}
    M.~Prince.
    \newblock {\em Does Active Learning Work? A Review of the Research}.
    \newblock Journal of Engineering Education, 2004.

  \bibitem{fabiane2024}
    F.~Santos et al.
    \newblock {\em Ensino Remoto: Desafios e Oportunidades}.
    \newblock Revista Brasileira de Educação, 2024.
    
  \beamertemplatearticlebibitems

  \bibitem{huang2021}
    J.~Huang et al.
    \newblock Chatbot Architecture and Design.
    \newblock {\em IEEE Transactions on Education}, 67(2):145--160,
    2021.
    
  \bibitem{okonkwo2021}
    C.~Okonkwo et al.
    \newblock Educational Bots in Remote Learning.
    \newblock {\em Computers \& Education}, 175:104--115,
    2021.
  \end{thebibliography}
\end{frame}

}

\end{document}


