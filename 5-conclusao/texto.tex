\chapter{Conclusão}
\label{cap:conclusao}

% Usar o graphicspath para buscar figuras no subdiretório figuras
\graphicspath{\currfiledir/figuras/}

%=====================================================

A implementação do bot desenvolvido com a biblioteca Concord demonstrou que é possível integrar, de forma natural e não invasiva, tecnologias de interação ativa ao ambiente de ensino remoto. A ferramenta permitiu que alunos interagissem com o conteúdo de aula e expressassem feedbacks espontâneos, contribuindo para a adaptação do professor em tempo real.

%=====================================================

\section{Síntese dos Resultados}
\label{sec:sintese}

% WIP

%=====================================================

\section{Limitações do Estudo}
\label{sec:limitacoes-conclusao}

A metodologia e implementação deste trabalho apresentam algumas limitações que devem ser consideradas na interpretação dos resultados e no planejamento de estudos futuros:

\begin{enumerate}
\item \textbf{Amostra limitada}: O número de participantes e disciplinas pode não representar adequadamente todos os contextos educacionais, restringindo a generalização dos resultados.
\item \textbf{Efeito novidade}: O interesse inicial pela tecnologia pode ter influenciado positivamente os resultados de curto prazo, um fenômeno comum em estudos de novas tecnologias educacionais.
\item \textbf{Viés de seleção}: Os participantes voluntários podem não representar adequadamente o perfil completo de alunos e professores, particularmente no que se refere a diferentes níveis de familiaridade tecnológica.
\item \textbf{Duração reduzida}: Enquanto um curso real se estende por semanas ou meses, o experimento teve duração limitada, impossibilitando a observação de efeitos de longo prazo como adaptação dos usuários, fadiga ou mudanças no engajamento ao longo do tempo.
\item \textbf{Diversidade contextual}: O experimento foi realizado em um contexto específico (disciplina de programação em nível de graduação), o que pode limitar a generalização dos resultados para outros contextos educacionais que possuem necessidades e dinâmicas distintas.
\item \textbf{Ambiente simulado}: Embora o experimento tenha buscado recriar condições próximas à realidade, o ambiente de sala de aula simulado pode não ter reproduzido completamente as dinâmicas e desafios de um curso real, como discutido na Seção \ref{subsec:desafios}. Os participantes estavam cientes de que estavam em um experimento, o que pode ter afetado seu comportamento (efeito Hawthorne).
\item \textbf{Avaliação de IHC}: Conforme discutido na Seção \ref{subsec:ihc}, interfaces centradas no usuário exigem avaliações extensivas. O experimento realizado pode não ter capturado todos os aspectos de usabilidade e experiência do usuário necessários para uma avaliação completa dos princípios de design centrado no usuário.
\item \textbf{Limitações técnicas}: A implementação em C usando a biblioteca Concord, embora tenha proporcionado controle granular sobre a API, pode apresentar desafios de manutenção e extensibilidade em comparação com linguagens de mais alto nível.
\end{enumerate}

Reconhecer estas limitações oferece perspectivas importantes para a interpretação dos resultados e para o direcionamento de pesquisas futuras neste campo.

%=====================================================

\section{Contribuições do Trabalho}
\label{sec:contribuicoes}

Este trabalho contribui para o campo da educação digital ao:

\begin{enumerate}
\item Demonstrar a viabilidade de integração sutil de tecnologias interativas no ensino remoto
\item Fornecer evidências empíricas sobre o impacto positivo de bots educacionais no engajamento
\item Propor uma arquitetura flexível e de baixo impacto para interações educacionais
\item Estabelecer métricas para avaliação da eficácia de ferramentas de interação em ambientes virtuais
\end{enumerate}

%=====================================================

\section{Trabalhos Futuros}
\label{sec:trabalhos-futuros}

Para trabalhos futuros, recomenda-se:

\begin{enumerate}
\item Expandir a aplicação do bot para outros contextos e disciplinas além da programação
\item Integrar suporte a outras plataformas educacionais (como Moodle, Google Meet, BBB (Big Blue Button), etc.)
\item Explorar o uso de inteligência artificial para tornar os feedbacks ainda mais personalizados e contextuais
\item Realizar estudos longitudinais para avaliar o impacto a longo prazo no desempenho acadêmico
\item Desenvolver recursos adicionais para análise de padrões de interação e geração de insights pedagógicos
\item Investigar o potencial do bot como ferramenta de avaliação contínua e formativa
\end{enumerate}

A promissora interseção entre tecnologia e educação continua a oferecer oportunidades para melhorar a experiência de ensino-aprendizagem, especialmente em contextos remotos, e o presente trabalho busca contribuir com este avanço.
